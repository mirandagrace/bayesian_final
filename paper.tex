\documentclass{article}\usepackage[]{graphicx}\usepackage[]{color}
%% maxwidth is the original width if it is less than linewidth
%% otherwise use linewidth (to make sure the graphics do not exceed the margin)
\makeatletter
\def\maxwidth{ %
  \ifdim\Gin@nat@width>\linewidth
    \linewidth
  \else
    \Gin@nat@width
  \fi
}
\makeatother

\definecolor{fgcolor}{rgb}{0.345, 0.345, 0.345}
\newcommand{\hlnum}[1]{\textcolor[rgb]{0.686,0.059,0.569}{#1}}%
\newcommand{\hlstr}[1]{\textcolor[rgb]{0.192,0.494,0.8}{#1}}%
\newcommand{\hlcom}[1]{\textcolor[rgb]{0.678,0.584,0.686}{\textit{#1}}}%
\newcommand{\hlopt}[1]{\textcolor[rgb]{0,0,0}{#1}}%
\newcommand{\hlstd}[1]{\textcolor[rgb]{0.345,0.345,0.345}{#1}}%
\newcommand{\hlkwa}[1]{\textcolor[rgb]{0.161,0.373,0.58}{\textbf{#1}}}%
\newcommand{\hlkwb}[1]{\textcolor[rgb]{0.69,0.353,0.396}{#1}}%
\newcommand{\hlkwc}[1]{\textcolor[rgb]{0.333,0.667,0.333}{#1}}%
\newcommand{\hlkwd}[1]{\textcolor[rgb]{0.737,0.353,0.396}{\textbf{#1}}}%

\usepackage{framed}
\makeatletter
\newenvironment{kframe}{%
 \def\at@end@of@kframe{}%
 \ifinner\ifhmode%
  \def\at@end@of@kframe{\end{minipage}}%
  \begin{minipage}{\columnwidth}%
 \fi\fi%
 \def\FrameCommand##1{\hskip\@totalleftmargin \hskip-\fboxsep
 \colorbox{shadecolor}{##1}\hskip-\fboxsep
     % There is no \\@totalrightmargin, so:
     \hskip-\linewidth \hskip-\@totalleftmargin \hskip\columnwidth}%
 \MakeFramed {\advance\hsize-\width
   \@totalleftmargin\z@ \linewidth\hsize
   \@setminipage}}%
 {\par\unskip\endMakeFramed%
 \at@end@of@kframe}
\makeatother

\definecolor{shadecolor}{rgb}{.97, .97, .97}
\definecolor{messagecolor}{rgb}{0, 0, 0}
\definecolor{warningcolor}{rgb}{1, 0, 1}
\definecolor{errorcolor}{rgb}{1, 0, 0}
\newenvironment{knitrout}{}{} % an empty environment to be redefined in TeX

\usepackage{alltt}
\usepackage[margin=0.75in]{geometry}
\usepackage{amsmath,amssymb}
\usepackage{graphicx,float}
\usepackage{multirow,setspace}
\usepackage{enumerate}
\usepackage{booktabs}
\usepackage{placeins}
\usepackage{cite}
\usepackage{caption}
\usepackage{subcaption}
\usepackage{wrapfig}
\usepackage{xcolor}
\usepackage{hyperref}
\usepackage{listings}

\newcommand{\HRule}{\rule{\linewidth}{0.5mm}}
\newcommand{\tab}{\hspace{0.5cm}}
\newcommand{\modref}[1]{(\ref{#1})}
\mathchardef\mhyphen="2D




\newcounter{DefnCounter}
\setcounter{DefnCounter}{1}

\newcounter{ThmCounter}
\setcounter{ThmCounter}{1}

\newcounter{ExampleCounter}
\setcounter{ExampleCounter}{1}

\newcommand{\defn}[1]{\textsc{Definition 1.\arabic{DefnCounter}\stepcounter{DefnCounter}: #1\\} }
\newcommand{\thm}{\textsc{Theorem 1.\arabic{ThmCounter}\stepcounter{ThmCounter}\\} }
\newcommand{\ex}{\textsc{Example 1.\arabic{ExampleCounter}\stepcounter{ExampleCounter}\\} }

\newcommand{\Defn}{\underline{Definition}}
\newcommand{\Q}{\underline{Question}}
\newcommand{\Qs}{\underline{Questions}}
\newcommand{\bbeta}{{\mbox{\boldmath$\beta$}}}
\newcommand{\bmu}{{\mbox{\boldmath$\mu$}}}
\newcommand{\balpha}{{\mbox{\boldmath$\alpha$}}}
\newcommand{\btheta}{{\mbox{\boldmath$\theta$}}}
\newcommand{\bphi}{{\mbox{\boldmath$\phi$}}}
\newcommand{\bSigma}{{\mbox{\boldmath$\Sigma$}}}
\newcommand{\bLambda}{{\mbox{\boldmath$\Lambda$}}}
\newcommand{\bpi}{{\mbox{\boldmath$\pi$}}}
\newcommand{\R}{\texttt{R}}
\newcommand{\Lik}{\mathcal{L}}
\newcommand{\bx}{\textbf{x}}
\newcommand{\by}{\textbf{y}}
\newcommand{\bX}{\textbf{X}}
\newcommand{\sic}{\text{Inv-}\chi^2}


\newcommand{\sao}{SaO$_2$}

\setlength{\marginparwidth}{2cm}
\IfFileExists{upquote.sty}{\usepackage{upquote}}{}
\begin{document}
\begin{center}
	\vspace{0.1cm}
	\textsc{\LARGE MATH 640 Final Project} \\[0.1cm]
	Jason Michaels (jam521), Niko Paulson (ndp32), Miranda Seitz-McLeese (mgs85) 
\end{center}
\section{Introduction}
\label{s:intro}
This analysis will be done on a data set of a variety of measurements about severe weather in the United States. 
The data set contains a variety of measures from severe weather events in the United States from 1996-2016. It was taken from the NOAA website \cite{dataset}.
For this project we focused on the deaths directly attributable to the event.
Understanding how and at what rate severe weather events become lethal in the United States has tremendous public health impacts.
In this paper we compare four possible models for the deaths: The traditional Poisson and negative binomial distributions, as well as the zero inflated variant of each.

The remainder of this analysis is organized as follows: Section~\ref{s:methods} discusses and derives the models. Section~\ref{s:results} describes the results of the analysis. And Section~\ref{s:discussion} contains the conclusions.

\section{Methods}
\label{s:methods}
The deaths attributed to a severe weather event is `count' data. 
The most common model used for count data is the Poisson distribution. 
However for some weather events, the negative binomial model is a better fit, because the Poisson distribution assumes that the events being counted occur independently.

Fortunately, the vast majority of severe weather events in the United States involve no deaths, therefore we wanted to also account for the possibility of structural zeros, therefore we also considered zero inflated variants. 
These distributions are created by returning $0$ with probability $\sigma$ and sampling from the original distribution with probability $(1-\sigma)$. 

We will derive and fit a model for each of the four distributions and see if there is a difference in our results and evaluate to determine which model best fits the data.

\subsection{Poisson}
\label{ss:mPoisson}

\subsection{Negative Binomial}
\label{ss:mNBinom}


\subsection{Zero Inflated Poisson}
\label{ss:mZiPoisson}
The Zero Inflated Poisson (ZIP) model has two parameters. The parameter p is the probability of a structural zero, and $\lambda$ corresponds to the parameter in a typical Poisson model. For a single observation x, the probability density is:

\[
p(x|p, \lambda) = pI_{x=0}(x) + (1-p)\frac{e^{-\lambda}\lambda^x}{x!}
\]


\noindent We can write the likelihood as follows:
$$
L(p, \lambda|X) = \prod_{x_i=0}\bigg[p+(1-p)\frac{e^{-\lambda}\lambda^{x_i}}{x_i!}\bigg]\prod_{x_i \ne 0}\bigg[(1-p)\frac{e^{-\lambda}\lambda^{x_i}}{x_i!}\bigg]
$$

\noindent Bayarri, Berger, and Datta (2008) \cite{ZIP} suggest using the prior distribution $\pi(\lambda, p) \propto \frac{1}{\sqrt{\lambda}}I(0<p<1)$. This gives us the following posterior 

\[
\prod_{x_i=0}\bigg[p+(1-p)\frac{e^{-\lambda}\lambda^{x_i}}{x_i!}\bigg]\prod_{x_i \ne 0}\bigg[(1-p)\frac{e^{-\lambda}\lambda^{x_i - 1/2}}{x_i!}\bigg]
\]

\noindent In obtaining our full conditionals, we can simplify this slightly to obtain the following:

\[
p(\lambda|X, p) \propto \prod_{x_i=0}\bigg[p+(1-p)\frac{e^{-\lambda}\lambda^{x_i}}{x_i!}\bigg]\prod_{x_i \ne 0}\bigg[e^{-\lambda}\lambda^{x_i - 1/2}\bigg]
\]

\[
p(p|X, \lambda) \propto \prod_{x_i=0}\bigg[p+(1-p)\frac{e^{-\lambda}\lambda^{x_i}}{x_i!}\bigg]\prod_{x_i \ne 0}\bigg[(1-p)\bigg]
\]

\noindent Neither of these distributions is recognizable. We can use a Metropolis-Hastings algorithm to sample from both of them. We will use a beta distribution as a proposal for p, and a gamma for $\lambda$. We will tune them to obtain a better acceptance rate. 

\subsection{Zero Inflated Negative Binomial}
\label{ss:mZiNBinom}
The Zero Inflated Negative Binomial (ZINB) model has three parameters $\sigma,$ the probability of a structural zero, and $p,r$ the usual negative binomial parameters. 
For a single $X$ the probability density is: 
$$p(X|\sigma, p, r) = \sigma I_{X=0}(X) + (1-\sigma)\frac{\Gamma(r+X)}{\Gamma(r)X!}.$$
We take the uniform priors for $\sigma$ and $p$ as well as the non-informative gamma for $r$ which is $r^{-1/2}$. For a full derivation, see \ref{a:dZINB}. My posterior is:
$$p(r,\sigma, p|X)\propto\left(\sigma + (1-\sigma)(1-p)^r\right)^Z(1-\sigma)^{N-Z}(1-p)^{(N-Z)r}p^{\sum_{i=1}^NX_i}r^{-1/2}\prod_{i=1}^N\left(\frac{\Gamma(r+X_i)}{\Gamma(r)}\right)$$
This distribution does not factor nicely, so I will use the Metropolis algorithm to sample from it. 
Because this posterior does not suggest any obvious proposal distributions I will sample each independently from a normal distribution centered at $\theta^*$, and with a variance that is tuned to yeild an appropriate acceptance rate.

\section{Results}
\label{s:results}
\subsection{Poisson}
\label{ss:rPoisson}

\subsection{Negative Binomial}
\label{ss:rNBinom}

\subsection{Zero Inflated Poisson}
\label{ss:rZiPoisson}

\subsection{Zero Inflated Negative Binomial}
\label{ss:rZiNBinom}

\section{Discussion}
\label{s:discussion}

\begin{thebibliography}{2}
  \bibitem{dataset} NOAA's Severe Weather Data Inventory, 
    \url{https://www1.ncdc.noaa.gov/pub/data/swdi/stormevents/csvfiles/}. 
    Accessed April 2017.
  \bibitem{ZIP} Bayarri, M., Berger, J., Datta, G. (2008). Objective testing of Poisson versus inflated Poisson models. IMS 	Collections, 3, 105-121. 
\end{thebibliography}
\clearpage
\appendix
\section{Code}
\label{a:code}
This appendix includes the code used to implement the models.

\section{Derivations}
\label{a:derivation}
This appendix will include details on the calculations required to derive our models.
\subsection{Zero Inflated Negative Binomial Derivation}
\label{a:dZINB}
The likelihood for the ZINB is 
\begin{align*}
\mathcal{L}(X|\sigma, p, r) &= \prod_{i=1}^N \sigma I_{X=0}(X_i) + (1-\sigma)\frac{\Gamma(r+X_i)}{\Gamma(r)X_i!}\\
\intertext{For ease of notation let $Z$ be the number of zero values in $X$, and $N$ be the total number of observations.}
&=\prod_{X_i=0}\left(\sigma + (1-\sigma)p^{X_i}(1-p)^r\frac{\Gamma(r+X_i)}{\Gamma(r)X_i!}\right)\prod_{X_i\ne 0}\left((1-\sigma)p^{X_i}(1-p)^r\frac{\Gamma(r+X_i)}{\Gamma(r)X_i!}\right)\\
&=\left(\sigma + (1-\sigma)(1-p)^r\right)^Z\prod_{X_i\ne 0}\left((1-\sigma)p^{X_i}(1-p)^r\frac{\Gamma(r+X_i)}{\Gamma(r)X_i!}\right)\\
&\propto\left(\sigma + (1-\sigma)(1-p)^r\right)^Z\prod_{X_i\ne 0}\left((1-\sigma)p^{X_i}(1-p)^r\frac{\Gamma(r+X_i)}{\Gamma(r)}\right)\\
&\propto\left(\sigma + (1-\sigma)(1-p)^r\right)^Z(1-\sigma)^{N-Z}(1-p)^{(N-Z)r}p^{\sum_{i=1}^NX_i}\prod_{i=1}^n\left(\frac{\Gamma(r+X_i)}{\Gamma(r)}\right)
\end{align*}
As mentioned in section \ref{ss:mZiNBinom} we take the uniform priors for $\sigma$ and $p$ as well as the non-informative gamma for $r$ which is $r^{-1/2}$. Therefore my joint posterior is:
\begin{align*}
p(r,\sigma, p|X)\propto&\left(\sigma + (1-\sigma)(1-p)^r\right)^Z(1-\sigma)^{N-Z}(1-p)^{(N-Z)r}p^{\sum_{i=1}^NX_i}r^{-1/2}\prod_{i=1}^N\left(\frac{\Gamma(r+X_i)}{\Gamma(r)}\right)\\
\intertext{I am now going to take the log of the posterior because it helps with computation}
\ln\left(p(r,\sigma, p|X)\right)\propto&Z\ln\left(1 + (1/\sigma-1)(1-p)^r\right) + Z\ln(\sigma) + (N-Z)\ln(1-\sigma)+(N-Z)r\ln(1-p)\\
&+\sum_{i=1}^NX_i\ln(p)-\ln(r)/2-N\ln(\Gamma(r))+\sum_{i=1}^N\ln(\Gamma(r+X_i))
\end{align*}

\end{document}
