%%%%%%%%%%%%%%%%%%%%%%%%%%%%%%%%%%%%%%%%%
% Beamer Presentation
% LaTeX Template
% Version 1.0 (10/11/12)
%
% This template has been downloaded from:
% http://www.LaTeXTemplates.com
%
% License:
% CC BY-NC-SA 3.0 (http://creativecommons.org/licenses/by-nc-sa/3.0/)
%
%%%%%%%%%%%%%%%%%%%%%%%%%%%%%%%%%%%%%%%%%

%----------------------------------------------------------------------------------------
%	PACKAGES AND THEMES
%----------------------------------------------------------------------------------------

\documentclass{beamer}

\mode<presentation> {
	
	% The Beamer class comes with a number of default slide themes
	% which change the colors and layouts of slides. Below this is a list
	% of all the themes, uncomment each in turn to see what they look like.
	
	\usetheme{default}
	%\usetheme{AnnArbor}
	%\usetheme{Antibes}
	%\usetheme{Bergen}
	%\usetheme{Berkeley}
	%\usetheme{Berlin}
	%\usetheme{Boadilla}
	%\usetheme{CambridgeUS}
	%\usetheme{Copenhagen}
	%\usetheme{Darmstadt}
	%\usetheme{Dresden}
	%\usetheme{Frankfurt}
	%\usetheme{Goettingen}
	%\usetheme{Hannover}
	%\usetheme{Ilmenau}
	%\usetheme{JuanLesPins}
	%\usetheme{Luebeck}
	%\usetheme{Madrid}
	%\usetheme{Malmoe}
	%\usetheme{Marburg}
	%\usetheme{Montpellier}
	%\usetheme{PaloAlto}
	%\usetheme{Pittsburgh}
	%\usetheme{Rochester}
	%\usetheme{Singapore}
	%\usetheme{Szeged}
	%\usetheme{Warsaw}
	
	% As well as themes, the Beamer class has a number of color themes
	% for any slide theme. Uncomment each of these in turn to see how it
	% changes the colors of your current slide theme.
	
	%\usecolortheme{albatross}
	\usecolortheme{beaver}
	%\usecolortheme{beetle}
	%\usecolortheme{crane}
	%\usecolortheme{dolphin}
	%\usecolortheme{dove}
	%\usecolortheme{fly}
	%\usecolortheme{lily}
	%\usecolortheme{orchid}
	%\usecolortheme{rose}
	%\usecolortheme{seagull}
	%\usecolortheme{seahorse}
	%\usecolortheme{whale}
	%\usecolortheme{wolverine}
	
	%\setbeamertemplate{footline} % To remove the footer line in all slides uncomment this line
	%\setbeamertemplate{footline}[page number] % To replace the footer line in all slides with a simple slide count uncomment this line
	
	%\setbeamertemplate{navigation symbols}{} % To remove the navigation symbols from the bottom of all slides uncomment this line
}

\usepackage{sidecap}%[capbesideposition=inside, facing=yes,capbesidesep=quad]
\usepackage{graphicx} % Allows including images
\usepackage{booktabs} % Allows the use of \toprule, \midrule and \bottomrule in tables
%----------------------------------------------------------------------------------------
%	TITLE PAGE
%----------------------------------------------------------------------------------------

\title[Four Models]{Four Models for Severe Weather Deaths} % The short title appears at the bottom of every slide, the full title is only on the title page

\author{Jason Michaels (jam521), Niko Paulson (ndp32), \\Miranda Seitz-McLeese (mgs85) } % Your name
%\institute[UCLA] % Your institution as it will appear on the bottom of every slide, may be shorthand to save space
%{
%	University of California \\ % Your institution for the title page
%	\medskip
%	\textit{john@smith.com} % Your email address
%}
\date{May 11, 2017} % Date, can be changed to a custom date

\begin{document}
	
	\begin{frame}
		\titlepage % Print the title page as the first slide
	\end{frame}
	
	\begin{frame}
		\frametitle{Overview} % Table of contents slide, comment this block out to remove it
		\tableofcontents % Throughout your presentation, if you choose to use \section{} and \subsection{} commands, these will automatically be printed on this slide as an overview of your presentation
	\end{frame}
	
	%----------------------------------------------------------------------------------------
	%	PRESENTATION SLIDES
	%----------------------------------------------------------------------------------------
	\section{Introduction} 
	\begin{frame}
		\frametitle{Introduction: Goals and Motivation}
	\end{frame}
	\begin{frame}
		\frametitle{Introduction: Data}
		The data for this project was taken from the NOAA severe weather events dataset. 
		In its original form this dataset includes a variety of measurements for all forms of severe weather in the United States dating from 1950. In the interest of time we restricted the dataset to:
		\begin{itemize}
			\item Deaths directly attributable to Tornadoes and Flash floods 
			\item Occurred between 1996-2016
		\end{itemize}
	\end{frame}
	\begin{frame}
		\frametitle{Introduction: Base Models}
		% description of negative binomials and poisson models
	\end{frame}
	\begin{frame}
		\frametitle{Introduction: Zero Inflated models}
		% overview of what a zero inflated models are 
	\end{frame}
	\begin{frame}
		\frametitle{Introduction: Four Models}
		% list the four models
	\end{frame}
	
	\section{Negative Binomial Model}
	\begin{frame}
		\frametitle{Negative Binomial}
		% Slide one
	\end{frame}
	\begin{frame}
		\frametitle{Negative Binomial}
		% Slide two
	\end{frame}
	
	\section{Poisson Model}
	\begin{frame}
		\frametitle{Poisson Model}
		% Slide one
	\end{frame}
	\begin{frame}
		\frametitle{Poisson Model}
		% Slide two
	\end{frame}
	
	\section{Zero Inflated Poisson Model}
	\begin{frame}
		\frametitle{Zero Inflated Poisson Model}
		% Slide one
	\end{frame}
	\begin{frame}
		\frametitle{Zero Inflated Poisson Model}
		% Slide two
	\end{frame}
	
	\section{Zero Inflated Negative Binomial Model}
	\begin{frame}
		\frametitle{Zero Inflated Negative Binomial Model}
		% Slide one
	\end{frame}
	\begin{frame}
		\frametitle{Zero Inflated Negative Binomial Model}
		% Slide two
	\end{frame}
	
	\section{Conclusions}
	\begin{frame}
		\frametitle{Conclusions: Model Evaluation Table}
		% Waic dic table goes here
	\end{frame}
	\begin{frame}
		\frametitle{Conclusions: Interpreting the Best Model}
		% Waic dic table goes here
	\end{frame}
	\begin{frame}
		\frametitle{Conclusions: Future Work}
		% List of things we might have done if we'd had time
		\begin{enumerate}
			\item{Re parameterize the ZINB model/look into other techniques.}
			\item{See if the results hold on other extreme weather events.}
		\end{enumerate}
	\end{frame}
	\begin{frame}
		\frametitle{Conclusions}
		Questions?
	\end{frame}
	
\end{document} 